\documentclass[12pt]{article}
\usepackage{authblk}
\usepackage{amsmath}
\usepackage{amssymb}

\title{\textbf{Prediction of Remaining Useful Life (RUL) of Batteries using ODE-RNNs}}
\author{Shubham Mate}

\author{Nandanavanam VS Sai Saketh}


\affil{Department of Mathematical Science, IIT (BHU), Varanasi}
\affil{\texttt{mate.shubhamvijay.mat23@itbhu.ac.in, nvssai.saketh.mat23@itbhu.ac.in}}

\date{November 2024}

\begin{document}
	
	\maketitle
	\section{Objective}
		
		\small{To predict the Remaining Useful Life (RUL) of a battery by giving 20 observations which consists of the various features of the batteries using a neural ODE without using the capacity (mAH) feature which is used in traditional methods. The model was to be trained on the given dataset.}
		\subsection{Dataset}
			\small{The dataset to be used is the Battery Remaining Useful Life Prediction dataset on Kaggle. The Hawaii Natural Energy Institute examined 14 NMC-LCO 18650 batteries with a nominal capacity of 2.8 Ah, which were cycled over 1000 times at 25°C with a CC-CV charge rate of C/2 rate and discharge rate of 1.5C. This data released by Hawaii Natural Energy Institute was used to make this dataset. It contains the following features:
			\begin{description}
				\item[Cycle Index] The Cycle number of current cycle
				\item[Discharge Time (s)] time that takes the voltage to reach its minimum value in one discharge cycle.
				\item[Decrement 3.6-3.4V (s)] It represents the time taken for voltage to drop from 3.6V to 3.4V during a discharge cycle.
				\item[Max. Voltage Dischar. (V)] The initial and maximum voltage in the discharging phase.
				\item[Min. Voltage Charg. (V)] The initial value of voltage when charging.
				\item[Time at 4.15V (s)] The time to reach 4.15V in charging phase.
				\item[Time constant current (s)] The time in which the current stays constant at its max. value.
				\item[Charging time (s)] The total time for charging.
				\item[RUL] The Remaining Useful Life of battery after every cycle.
			\end{description}}
	\section{\large{Mathematical Formulation and Explaination}}
		\small{In this section, we'll look at and recap the mathematical formulation of the various techniques and components used in the model}
		\subsection{Neural ODE}
			\small{Numerous Models build complicated transformations by composing a sequence of
			transformations to the input or a maintained hidden state. These transformations are mathematically expressed as:
			\begin{equation*}
				\mathbf{z}_{t+1} = \mathbf{z}_{t} + f(\mathbf{z}_{t}, \theta_{t})
			\end{equation*}
			where $t \in \{0, \dots, T \}$ and $h_{t} \in \mathbb{R}$. These iterative
			updates can be seen as an Euler discretization of a continuous transformation. Neural ODE aims to parameterize the continuous dynamics of $z_{t}$ given by the equation:
			\begin{equation*}
				\frac{d\mathbf{z}(t)}{dt} = f(\mathbf{z}(t), t, \theta) 
			\end{equation*}
			The output of Neural ODE is generated with a black box differential equation solver. Therefore, the $\mathbf{z}(T)$ can be obtained by:
			\begin{equation*}
				\mathbf{z}(T) = \mathbf{z}(t_0) + \int_{t_{0}}^T \mathbf{f}(\mathbf{z}(t), t; \mathbf{\theta}) \, dt = \text{ODESolve}(\mathbf{z}(t_0), f, t_0, T, \theta)
			\end{equation*}
			In order to avoid backpropogating through the black box differential solver, Neural ODE uses adjoint sensitivity method to calculate required gradients. At every point, it maintains an adjoint state $\mathbf{a}(t)$ whose dynamics are given by another ODE:
			\begin{equation*}
				\frac{d\mathbf{a}(t)}{dt} = -\mathbf{a}(t)^T \frac{\partial \mathbf{f}(\mathbf{z}(t), t; \mathbf{\theta})}{\partial \mathbf{z}(t)}
			\end{equation*}
			Another ODE solver is used to solve the dynamics of $\mathbf{a}(t)$.
			The Gradient $\frac{d\mathbf{L}}{d\theta}$ is finally computed by the equation:
			\begin{equation*}
				\frac{d\mathcal{L}}{d\mathbf{\theta}} = \int_0^T \mathbf{a}(t)^T \frac{\partial \mathbf{f}(\mathbf{z}(t), t; \mathbf{\theta})}{\partial \mathbf{\theta}} dt
			\end{equation*}
			We use Neural ODEs to model the dynamics of the hidden state of a RNN.}
		\subsection{Neural ODE based GRU}
			\footnotesize{RNN (Recurrent Neural Network) is a type of neural network which maintains a hidden state $\mathbf{h}_t$ which gets updated each time a new input is fed into it. This allows it to remember information from previous inputs. This makes them suitable to model sequential/time-series data in which current input may be dependant on previous inputs.
			\\
			GRU (Gated Recurrent Unit) is a variation of RNNs where each unit has an update gate ($z_t$) and reset gate ($r_t$) which allows them to select what information should be maintained by hidden states and what information should be forgotten. The equations for the GRU are:
			\begin{equation}
				z_t = \sigma(\mathbf{W}_z \mathbf{x}_t + \mathbf{U}_z \mathbf{h}_{t-1} + \mathbf{b}_z)
			\end{equation}
			\begin{equation}
				r_t = \sigma(\mathbf{W}_r \mathbf{x}_t + \mathbf{U}_r \mathbf{h}_{t-1} + \mathbf{b}_r)
			\end{equation}
			\begin{equation}
				\tilde{\mathbf{h}}_t = \tanh(\mathbf{W}_h \mathbf{x}_t + \mathbf{U}_h (r_t \odot \mathbf{h}_{t-1}) + \mathbf{b}_h)
			\end{equation}
			\begin{equation}
				\mathbf{h}_t = (1 - z_t) \odot \mathbf{h}_{t-1} + z_t \odot \tilde{\mathbf{h}}_t
			\end{equation}
			To incorporate a Neural ODE to update the hidden state, we give the following method:
			\\
			Suppose we have two inputs $\mathbf{x}_{t_1}$ and $\mathbf{x}_{t_2}$. Let us define a function UpdateGRU($\mathbf{x}$, $\mathbf{h}_t$) which takes an input and current hidden state and updates the states according to the equations given above. We also define a function $f$ (a neural network) with parameters $\theta$
			\\
			We use the following method to get the output:
			\begin{equation*}
				\mathbf{h}_{t_1} = \mathbf{UpdateGRU}(\mathbf{x}_{t_1}, \mathbf{h}_{t_0})
			\end{equation*}
			\begin{equation*}
				\mathbf{h}_{t_1} = \mathbf{ODESolve}(\mathbf{h}_{t_1}, f, t_1, t_2, \theta)
			\end{equation*}
			\begin{equation*}
				\mathbf{h}_{t_2} = \mathbf{UpdateGRU}(\mathbf{x}_{t_2}, \mathbf{h}_{t_1})
			\end{equation*}
			This helps us model the continuous dynamics of the hidden state of GRU in between the updates, which can help us even model time series data with irregular time interval.}
			
			
	

\end{document}